%%%%%%%%%%%%%%%%%%%%%%%%%%%%%%%%%%%%%%%%%%
%
% Cosecivi 2017
% http://gaia.fdi.ucm.es/sites/cosecivi17
% 12 pages max
%
%%%%%%%%%%%%%%%%%%%%%%%%%%%%%%%%%%%%%%%%%%

\documentclass{llncs}

\usepackage[english]{babel}
\usepackage[utf8]{inputenc}

\usepackage{epsfig}
\usepackage{graphicx}
\usepackage{subcaption}
\captionsetup{compatibility=false}
\usepackage{color}
\usepackage{amsfonts}
\usepackage{amsmath}
%\usepackage{mathabx}
\usepackage{hyperref}

%\usepackage{hyperref}
%\usepackage{subfigure}
%\usepackage[colorinlistoftodos, textwidth=3.2cm, shadow]{todonotes}

%\usepackage{algorithm}
%\usepackage{fixltx2e}
%\usepackage{algpseudocode}

%\usepackage{multirow}

% To use Call inside another Call (algorithms)
%\MakeRobust{\Call}


%%%%%%%%%%%%%%%%%%%%%%%%%%%%%%%%%%%%%%%%%%
%
% Title
%
%%%%%%%%%%%%%%%%%%%%%%%%%%%%%%%%%%%%%%%%%%
\title{Title\thanks{Supported by Spanish Ministry of Economy and Competitiveness under grant TIN2014-55006-R {\color{red} Kiko, ¿quieres añadir algo aquí?}}
}

\author{Nombre 1, Nombre 2, ..., Antonio A. S\'{a}nchez-Ruiz {\color{red} En el orden que queráis}}

\institute{
	Dep. Ingenier\'{\i}a del Software e Inteligencia Artificial \\
	Universidad Complutense de Madrid (Spain) \\
	\email{correo1, correo2, ..., antsanch@ucm.es}
}

\begin{document}

\maketitle

%%%%%%%%%%%%%%%%%%%%%%%%%%%%%%%%%%%%%%%%%%
%
% Abstract
%
%%%%%%%%%%%%%%%%%%%%%%%%%%%%%%%%%%%%%%%%%%
\begin{abstract}
Bla bla bla ... 
Dejar para el final

\keywords{keyword1, keyword2, ...}
\end{abstract}

%%%%%%%%%%%%%%%%%%%%%%%%%%%%%%%%%%%%%%%%%%
%
\section{Introduction}
\label{sec:intro}
%
%%%%%%%%%%%%%%%%%%%%%%%%%%%%%%%%%%%%%%%%%%

Dejar para el final

%%%%%%%%%%%%%%%%%%%%%%%%%%%%%%%%%%%%%%%%%%
%
\section{Ms. Pac-Man vs. Ghosts AI}
\label{sec:pacman}
%
%%%%%%%%%%%%%%%%%%%%%%%%%%%%%%%%%%%%%%%%%%

% Ejemplo de cómo se introduce una imagen
%
%\begin{figure}[tb]
%	\centering
%	\includegraphics[width=0.35\textwidth]{images/tetris.png}
%	\caption{A screen of the Tetris game.}
%	\label{fig:tetris}
%\end{figure}


\begin{itemize}
\item Cómo funciona el juego (imagen)
\item Arquitectura para creat bots
\item Competiciones en las que se ha usado
\end{itemize}


%%%%%%%%%%%%%%%%%%%%%%%%%%%%%%%%%%%%%%%%%%
%
\section{A bot based on grammatical evolution}
\label{sec:sec1}
%
%%%%%%%%%%%%%%%%%%%%%%%%%%%%%%%%%%%%%%%%%%

\begin{itemize}
\item Idea general
\item Parámetros que definen el estado de juego
\item Acciones
\item Gramática
\item Framework utilizado
\item Operadores de cruce y mutación
\item Resultados (¿comparación con otros bots?, ¿estrategias aprendidas?)
\end{itemize}


%%%%%%%%%%%%%%%%%%%%%%%%%%%%%%%%%%%%%%%%%%
%
\section{Multi-objective optimization}
\label{sec:sec3}
%
%%%%%%%%%%%%%%%%%%%%%%%%%%%%%%%%%%%%%%%%%%

\begin{itemize}
\item En qué consiste el multi-objetivo
\item por qué es interesante considerar varios objetivos en el juego
\item Experimentos y resultados (¿comparación con otros bots?, ¿estrategias aprendidas?)
\end{itemize}

%%%%%%%%%%%%%%%%%%%%%%%%%%%%%%%%%%%%%%%%%%
%
\section{Related Work}
\label{sec:relatedWork}
%
%%%%%%%%%%%%%%%%%%%%%%%%%%%%%%%%%%%%%%%%%%

\begin{itemize}
\item Sobre IA (otras técnicas) y pac-man
\item Sobre genéticos, evolutivos y juegos
\item Sobre Multi-objetivo y juegos 
\item Sobre operadores u otras cosas que hayáis utilizado
\item ...
\end{itemize}


%%%%%%%%%%%%%%%%%%%%%%%%%%%%%%%%%%%%%%%%%%
%
\section{Conclusions}
\label{sec:conclusions}
%
%%%%%%%%%%%%%%%%%%%%%%%%%%%%%%%%%%%%%%%%%%


Dejar para el final

Añadir al final la URL al repositorio de GitHub

\bibliographystyle{splncs}
\bibliography{references}

\end{document}



